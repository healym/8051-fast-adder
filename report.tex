\documentclass[journal]{IEEEtran}
\usepackage{graphics}

\begin{document}


\title{Implementation of a Carry Lookahead Fast Adder}
\author{
	Healy, Matthew
	\texttt{mhealy@mst.edu}\\
	\and
	Johnston, Jaxson
	\texttt{jnjt37@mst.edu}\\
	\and
	Grbe\`sa, Lukas
	\texttt{lgqq3@mst.edu}\
}

\maketitle


\begin{abstract}
This is the abstract. You can use this file to start your own LaTeX file,
and just delete the stuff you do not need. \LaTeX  is a lot like working
with HTML: you can specify where text effects begin, and where they end.
\end{abstract}

\section{Introduction}\label{sec:intro}

\section{Source Code}\label{sec:code}

I like using the \verb"verbatim" specification for computer code.
For example, here is something that appears in several
languages:

\begin{verbatim}
        MOV R3, #40H   ; #10H bit
        MOV R6, #48H   ; #40H bit
        MOV R7, #50H   ; #70H bit
        MOV R5, R2     ; length of operands stored in R2

  LOAD: MOV R4, @R6    ; temp hold for byte of R6 data
        XRL @R6, @R7   ; propagate
        ANL @R7, R4    ; generate
        INC R6         ; move to next bit of P
        INC R7         ; move to next bit of G
        DNJZ R5, LOAD

        MOV R5, R2     ; reset R5
        MUL R5, #8H    ; switch to bit counter
        MOV R5, A      ; load counter with length*8
        CLR C          ; C will be used as Ci in boolean equation
        MOV R6, #040H  ; bit addr
        MOV R7, #070H  ; bit addr

 CARRY: ANL C, @R6     ; intermediate = Ci AND P(i+1)
        ORL C, @R7     ; C(i+1) = intermediate OR G(i+1)
        MOV @R3, C     ; store C(i+1) in carry string
        INC R3         ; move to next bit of carry string
        INC R6         ; move to next bit of P
        INC R7         ; move to next bit of G
        DJNZ R5, CARRY

        MOV R3, #40H   ; reload R3 with start of carry string
        MOV R6, #48H   ; reload R6 with start of P
        MOV R5, R2     ; reload counter with length

   SUM: XRL @R6 @R3    ; compute final sum
        INC R6         ; move to next bit of P
        INC R3         ; move to next bit of Carry string
        DJNZ R5, SUM

        MOV R6, #48H   ; reset R6 to beginning of result string
        RET            ; result string pointed to by R6
\end{verbatim}


% Now here is the reference section.

\begin{thebibliography}{99}

  % Book
  \bibitem{Weste93} Neil H. E. Weste and Kamran Eshraghian, {\it Principles
  of CMOS VLSI Design}, 2nd ed. Reading, MA: Addison-Wesley, 1993.

  %Example of a Conference Paper
  \bibitem{LiY88} R. A. Lincoln and K. Yao, ``Efficient Systolic Kalman
  Filtering Design by Dependence Graph Mapping,'' in {\it VLSI Signal
  Processing, III}, IEEE Press, R. W. Brodersen and H. S. Moscovitz Eds.,
  1988, pp.~396--410.

  % Example of a Journal Paper
  \bibitem{BiS92} C. H. Bischof and G. M. Shroff, ``On Updating Signal
  Subspaces,'' {\it IEEE Trans. on Signal Processing}, vol.~40, no.~1,
  pp.~96--105, Jan. 1992.

\end{thebibliography}
\end{document}
