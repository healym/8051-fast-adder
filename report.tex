\documentclass[final]{ieee}
\usepackage{graphics}

\begin{document}


\title[journal_ex]{Implementation of a Carry Lookahead Fast Adder}
\author[mcw]{
Matthew Healy\member{Member} \\
\authorinfo{
     email: \mbox{mhealy@mst.edu}}
}
\author[mcw]{
Jaxson Johnston\member{Member} \\
 \authorinfo{
	  email: \mbox{mhealy@mst.edu}}
}
\author[mcw]{
Lukas Grbesa\member{Member} \\
 \authorinfo{
	  email: \mbox{mhealy@mst.edu}}
}



\maketitle


\begin{abstract}
This is the abstract. You can use this file to start your own LaTeX file,
and just delete the stuff you do not need. \LaTeX  is a lot like working
with HTML: you can specify where text effects begin, and where they end.
\end{abstract}

\section{Introduction}\label{sec:intro}
Here is the introduction.
Since there is no blank line between these first 3 sentences, they are
treated as one paragraph.
Here is a vertical space (of 0.3 inches):
\vspace{.3in}

And here is a \hspace{.3in}horizontal space (of 0.3 inches).

A blank line means that the last paragraph is over, and it is time to start
a new one.

You can have text in {\it italics} font, or in {\bf bold} font,
\underline{underlined}, and even $\overline{overlined}$.
What if you want overlined text, without italics? This can
be done by using \verb"mathrm" in the
$\overline{\mathrm{overlined}}$ specification.

Citing a reference: This is a book about VLSI \cite{Weste93}.
Also, the references contain a good conference paper \cite{LiY88},
and a good journal article \cite{BiS92}.


\begin{table}[!hbt]
\begin{center}
  \caption{Table captions go above tables.}
  \vspace{0.2in}
  \begin{tabular}{|r|c|c|c|}
     \hline
 & runs & hits & errors  \\
     \hline
Cardinals  & 2 & 2 & 1  \\
Panthers & 4 & 8 & 0  \\
Tigers  & 2 & 3 & 2  \\
Braves  & 3 & 10 & 3  \\
     \hline
  \end{tabular}
  \label{tab:example_tab}
\end{center}
\end{table}

What if you want to include a figure?
Here is an example, figure~\ref{fig:phasor1}, that is saved in
encapsulated postscript format.

\begin{figure}[!hbt]
  \centering
    \scalebox{.9}{\includegraphics{phasor1.eps}}
  \caption{Here is an example vector.}
  \label{fig:phasor1}
\end{figure}


Skip a lot of space  \bigskip  vertically.

\section{Here is some Computer Code}\label{sec:code}

I like using the \verb"verbatim" specification for computer code.
For example, here is something that appears in several
languages:

\begin{verbatim}
        MOV R3, #40H   ; #10H bit
        MOV R6, #48H   ; #40H bit
        MOV R7, #50H   ; #70H bit
        MOV R5, R2     ; length of operands stored in R2

  LOAD: MOV R4, @R6    ; temp hold for byte of R6 data
        XRL @R6, @R7   ; propagate
        ANL @R7, R4    ; generate
        INC R6         ; move to next bit of P
        INC R7         ; move to next bit of G
        DNJZ R5, LOAD

        MOV R5, R2     ; reset R5
        MUL R5, #8H    ; switch to bit counter
        MOV R5, A      ; load counter with length*8
        CLR C          ; C will be used as Ci in boolean equation
        MOV R6, #040H  ; bit addr
        MOV R7, #070H  ; bit addr

 CARRY: ANL C, @R6     ; intermediate = Ci AND P(i+1)
        ORL C, @R7     ; C(i+1) = intermediate OR G(i+1)
        MOV @R3, C     ; store C(i+1) in carry string
        INC R3         ; move to next bit of carry string
        INC R6         ; move to next bit of P
        INC R7         ; move to next bit of G
        DJNZ R5, CARRY

        MOV R3, #40H   ; reload R3 with start of carry string
        MOV R6, #48H   ; reload R6 with start of P
        MOV R5, R2     ; reload counter with length

   SUM: XRL @R6 @R3    ; compute final sum
        INC R6         ; move to next bit of P
        INC R3         ; move to next bit of Carry string
        DJNZ R5, SUM

        MOV R6, #48H   ; reset R6 to beginning of result string
        RET            ; result string pointed to by R6

\end{verbatim}

See how it makes the code stand out? I think it makes it
much easier to read, too.

\section{Filler}

It is always a good idea to have text between a section header and
a subheading.

\subsection{Sentences About Nothing}

This sentence does not really say anything important, but it does take up space.
This sentence does not really say anything important, but it does take up space.
This sentence does not really say anything important, but it does take up space.
This sentence does not really say anything important, but it does take up space.
This sentence does not really say anything important, but it does take up space.

This sentence does not really say anything important, but it does take up space.
This sentence does not really say anything important, but it does take up space.
This sentence does not really say anything important, but it does take up space.
This sentence does not really say anything important, but it does take up space.
This sentence does not really say anything important, but it does take up space.
This sentence does not really say anything important, but it does take up space.
This sentence does not really say anything important, but it does take up space.
This sentence does not really say anything important, but it does take up space.
{\bf I inserted a ``newpage'' command here to balance the columns.}

\newpage

\subsection{More About Nothing}

This sentence does not really say anything important, but it does take up space.
This sentence does not really say anything important, but it does take up space.
This sentence does not really say anything important, but it does take up space.
This sentence does not really say anything important, but it does take up space.
This sentence does not really say anything important, but it does take up space.
This sentence does not really say anything important, but it does take up space.
This sentence does not really say anything important, but it does take up space.
This sentence does not really say anything important, but it does take up space.
This sentence does not really say anything important, but it does take up space.
This sentence does not really say anything important, but it does take up space.

This sentence does not really say anything important, but it does take up space.
This sentence does not really say anything important, but it does take up space.




% Now here is the reference section.

\begin{thebibliography}{99}

  % Book
  \bibitem{Weste93} Neil H. E. Weste and Kamran Eshraghian, {\it Principles
  of CMOS VLSI Design}, 2nd ed. Reading, MA: Addison-Wesley, 1993.

  %Example of a Conference Paper
  \bibitem{LiY88} R. A. Lincoln and K. Yao, ``Efficient Systolic Kalman
  Filtering Design by Dependence Graph Mapping,'' in {\it VLSI Signal
  Processing, III}, IEEE Press, R. W. Brodersen and H. S. Moscovitz Eds.,
  1988, pp.~396--410.

  % Example of a Journal Paper
  \bibitem{BiS92} C. H. Bischof and G. M. Shroff, ``On Updating Signal
  Subspaces,'' {\it IEEE Trans. on Signal Processing}, vol.~40, no.~1,
  pp.~96--105, Jan. 1992.

\end{thebibliography}
\end{document}
